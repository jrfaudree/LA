\documentclass[11pt,fleqn]{article} 
\usepackage[margin=0.8in, head=0.8in]{geometry} 
\usepackage{amsmath, amssymb, amsthm}
\usepackage{fancyhdr} 
\usepackage{palatino, url, multicol}
\usepackage{graphicx} 
\usepackage[all]{xy}
\usepackage{polynom} 
\usepackage{pdfsync}
\usepackage{enumerate}
\usepackage{framed}
\usepackage{setspace}
\usepackage{array,tikz}
\pagestyle{fancy} 
\lhead{Linear Algebra}
\chead{Quiz 2}
\rhead{\Large{Name: \underline{\hspace{2in}}}}

\begin{document}
\renewcommand{\headrulewidth}{0pt}
\newcommand{\blank}[1]{\rule{#1}{0.75pt}}
\renewcommand{\d}{\displaystyle}
%\vspace*{-0.7in}
This quiz is worth 10 points.
\begin{enumerate}
\item (8 points) Determine whether each of the following scalar-valued functions of $n$-vectors is linear. If it is a linear function, give its inner product representation (i.e. an $n$ vector $a$ for which $f(x)=a^Tx$ for all $x$). If it is not linear, give specific $x$, $y$, $\alpha$, and $\beta$ for which superposition fails: $$( i.e. \; f(\alpha x + \beta y) \not = \alpha f(x) + \beta f(y)).$$
\begin{enumerate}
\item $f(x)$ is the average of the first 3 entries of vector $x.$ You an assume $n\geq 3.$

\vfill

\item $f(x)$ is minimum entry of $x$. That is $f(x)= \min \{x_1,x_2,x_3, \cdots x_n\}.$

\vfill
\end{enumerate}

\item (2 points) Suppose $f: \mathbb{R}^3 \to \mathbb{R}$ is a \textbf{linear function}. Further, suppose, $f(2,-4,3)=10$ and $f(2,1,0)=8.$ Determine the value of $f(2,-9,6)$ if possible. If this is not possible, explain why.
\vspace{3in}
\end{enumerate}
\end{document}