\documentclass[11pt,fleqn]{article} 
\usepackage[margin=0.8in, head=0.8in]{geometry} 
\usepackage{amsmath, amssymb, amsthm}
\usepackage{fancyhdr} 
\usepackage{palatino, url, multicol}
\usepackage{graphicx} 
\usepackage[all]{xy}
\usepackage{polynom} 
\usepackage{pdfsync}
\usepackage{enumerate}
\usepackage{framed}
\usepackage{setspace}
\usepackage{array,tikz,systeme}

\newcommand{\bbm}{\begin{bmatrix}}
\newcommand{\ebm}{\end{bmatrix}}

\newcommand{\Reals}{\mathbb{R}}
\def\vectwo#1#2{\begin{bmatrix}#1\\#2\end{bmatrix}}
\def\vecthree#1#2#3{\begin{bmatrix}#1\\#2\\#3\end{bmatrix}}
\def\vecfour#1#2#3#4{\begin{bmatrix}#1\\#2\\#3\\#4\end{bmatrix}}

\pagestyle{fancy} 
\lhead{Linear Algebra}
\chead{Quiz 7}
\rhead{}

\begin{document}
\renewcommand{\headrulewidth}{0pt}
\newcommand{\blank}[1]{\rule{#1}{0.75pt}}
\renewcommand{\d}{\displaystyle}
%\vspace*{-0.7in}
This quiz is worth 10 points. \hfill {\Large{Name: \underline{\hspace{2in}}}}
\begin{enumerate}
\item (6 points) Let $A=\bbm 1&2\\0&-1\\3&1 \ebm$ and $B=\bbm -2&1\\4&2 \ebm.$ Evaluate each expression below or state that the expression is not defined.\\
	\begin{enumerate}
	\item $2AB$
	\vfill
	\item $2BA$
	\vfill
	\item $A^2$
	\vfill
	\item $AA^T$
	\vfill
	\end{enumerate}
\newpage
\item (4 points) Let $C$ be an $m \times n$ matrix where $C_{ij}=\begin{cases} 1 & \text{student } i \text{ is in class } j \\
0 &  \text{student } i \text{ is not in class } j \end{cases}.$ Thus, the $m$ rows of matrix $C$ represent $m$ students and the $n$ columns of $C$ represent $n$ classes. \\
	\begin{enumerate}
	\item Let $A=CC^T.$
		\begin{enumerate}
		\item What are the dimensions of $A$?
		\vspace{.5in}
		\item Suppose $A_{34}=2.$ Write a sentence explaining what this means in terms of students and classes.
		\vfill
		\end{enumerate}
	\item Let $B=C^TC.$
		\begin{enumerate}
		\item What are the dimensions of $B$?
		\vspace{.5in}
		\item Suppose $B_{34}=2.$ Write a sentence explaining what this means in terms of students and classes.
		\vfill
		\end{enumerate}
	\end{enumerate}	
\end{enumerate}
\end{document}