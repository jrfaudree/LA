\documentclass[11pt,fleqn]{article} 
\usepackage[margin=0.8in, head=0.8in]{geometry} 
\usepackage{amsmath, amssymb, amsthm}
\usepackage{fancyhdr} 
\usepackage{palatino, url, multicol}
\usepackage{graphicx,tabularx,systeme} 
\usepackage[all]{xy}
\usepackage{polynom} 
\usepackage{pdfsync}
\usepackage{enumerate}
\usepackage{framed}
\usepackage{setspace}
\usepackage{array,tikz}

\newcommand{\bbm}{\begin{bmatrix}}
\newcommand{\ebm}{\end{bmatrix}}

\newcommand{\Reals}{\mathbb{R}}
\def\vectwo#1#2{\begin{bmatrix}#1\\#2\end{bmatrix}}
\def\vecthree#1#2#3{\begin{bmatrix}#1\\#2\\#3\end{bmatrix}}
\def\vecfour#1#2#3#4{\begin{bmatrix}#1\\#2\\#3\\#4\end{bmatrix}}

%\pagestyle{fancy} 
\lfoot{Linear}
\rfoot{Ch 6}
\renewcommand{\familydefault}{\sfdefault}
\begin{document}
{\emph{\fontsize{26}{28}\selectfont Fall 2024 \hfill
%{\fontsize{32}{36}\selectfont Calculus 1: Midterm 1}
\hfill Math F314X}}

\begin{center}
{\emph{%\fontsize{26}{28}\selectfont Spring 2024 
%%\hfill
{\fontsize{32}{36}\selectfont Linear Algebra: Midterm 1}
%%\hfill Math F251X}
}}
\end{center}

\vfill

\quad \hfill {\Large{Name: \underline{\hspace{2in}}}}

\vfill

{\fontsize{18}{22}\selectfont\emph{Rules:}}

\begin{itemize}
\item Show your work.

\item You may have a single handwritten sheet of paper with writing on one side.

\item You may use a calculatior 

\end{itemize}


\vfill

\def\emptybox{\hbox to 2em{\vrule height 16pt depth 8pt width 0pt\hfil}}
\def\tline{\noalign{\hrule}}
\centerline{\vbox{\offinterlineskip
{
\bf\sf\fontsize{18pt}{22pt}\selectfont
\hrule
\halign{
\vrule#&\strut\quad\hfil#\hfil\quad&\vrule#&\quad\hfil#\hfil\quad
&\vrule#&\quad\hfil#\hfil\quad&\vrule#\cr
height 3pt&\omit&&\omit&&\omit&\cr
&Problem&&Possible&&Score&\cr\tline
height 3pt&\omit&&\omit&&\omit&\cr
&1&&10&&\emptybox&\cr\tline
&2&&20&&\emptybox&\cr\tline
&3&&15&&\emptybox&\cr\tline
&4&&25&&\emptybox&\cr\tline
&5&&30&&\emptybox&\cr\tline
&Total&&100&&\emptybox&\cr
}\hrule}}}

\vfill

\newpage

\quad

\newpage
\begin{enumerate}

\item (10 points) Let $a$ and $b$ be $n$-vectors. Show by direct computation that $\Vert a+b \Vert^2 - \Vert a-b \Vert^2 = 4a^Tb.$
\vfill


\item (20 points) Let $a_1=( \frac{1}{2}, \frac{\sqrt{3}}{2})$ and $a_2=(\frac{-\sqrt{3}}{2}, \frac{1}{2})$.
	\begin{enumerate}
	\item Show that $a_1$ and $a_2$ are an \textbf{orthonormal} set of vectors.
	\vfill
	\item Write the vector $v=\vectwo{4}{-8}$ as a linear combination of $a_1$ and $a_2.$ Show your work.
	\vfill
	\end{enumerate}
\newpage

\item (15 points) Given the vectors $x_1=\vecthree 1 3 0, x_2=\vecthree 2 2 0, x_3=\vecthree 1 0 0, x_4=\vecthree 0 0 1.$
	\begin{enumerate}
	\item Carefully explain why no computation is necessary to determine that the vectors $x_1,x_2,x_3,$ and $x_4$ are linearly dependent.
	\vspace{2 in}
	\item Use the definition to show the vectors $x_1,x_2,x_3, x_4$  are linearly {dependent}.
	\vfill
%	\item Show that the vector $y=\vecthree 1 3 1$ can be written as a linear combination of the $x_i$'s in two different ways.
%	\vfill
	\end{enumerate}
\newpage
\item (25 points) Let $T= \left\{ v_1=\vecthree 1 2 2, v_2=\vecthree {-1} 0 0, v_3= \vecthree 0 3 4 \right\}.$
	\begin{enumerate}
	\item Show that $T$ is linearly independent.
	\vfill
\newpage
	\item Begin the Gram-Schmidt algorithm on the vectors $v_1,v_2,v_3,$ in the given order.\\
Recall $T= \left\{ v_1=\vecthree 1 2 2, v_2=\vecthree {-1} 0 0, v_3= \vecthree 0 3 4 \right\}.$
		\begin{enumerate}
		\item Find $\overline{q_1}$ and $q_1.$
		\vspace{2in}\
		\item Find $\overline{q_2}$ and $q_2.$
		\vfill
		\end{enumerate}
	\end{enumerate}
\newpage
\item (30 points) Short Answer
	\begin{enumerate}
	\item  Let $a_1=\vecfour 1 2 {-3} 4$ and $a_2=\vecfour 5 4 3 c$ where $c$ is some real number. Determine the values of $c$ such that the angle between $a_1$ and $a_2$ be obtuse. (An angle is obtuse if it is larger than 90 degrees.)
	\vfill
	\item If $\Vert x \Vert=10$ and $\Vert y \Vert =5,$ what are possible values for $\Vert x+y \Vert$?
	\vfill
	\item The vectors $v_1,v_2,\cdots,v_k$ are $k$- vectors and they form an orthonormal set of vectors. Can you conclude that they are a \textbf{basis}? \textbf{Justify} your conclusion using complete sentences.
	\vfill
	\newpage
	\item Suppose $f(x)$ is a scalar-valued function of a $50$-vector that outputs the sum the first 30 values in the input vector $x.$ Determine whether $f(x)$ is a linear function and demonstrate your answer is correct.
	\vfill
	\item Determine a matrix $A$ such that \; $A\bbm x_1\\x_2\\x_3\\x_4\\x_5\\ \ebm = \bbm x_5 \\x_4\\x_3 \ebm$
	\vfill
	\end{enumerate}
\end{enumerate}
\end{document}

%%%%ENDDOCUMENT


