\documentclass[11pt,fleqn]{article} 
\usepackage[margin=0.8in, head=0.8in]{geometry} 
\usepackage{amsmath, amssymb, amsthm}
\usepackage{fancyhdr} 
\usepackage{palatino, url, multicol}
\usepackage{graphicx,tabularx,systeme} 
\usepackage[all]{xy}
\usepackage{polynom} 
\usepackage{pdfsync}
\usepackage{enumerate}
\usepackage{framed}
\usepackage{setspace}
\usepackage{array,tikz}

\newcommand{\bbm}{\begin{bmatrix}}
\newcommand{\ebm}{\end{bmatrix}}

\newcommand{\Reals}{\mathbb{R}}
\def\vectwo#1#2{\begin{bmatrix}#1\\#2\end{bmatrix}}
\def\vecthree#1#2#3{\begin{bmatrix}#1\\#2\\#3\end{bmatrix}}
\def\vecfour#1#2#3#4{\begin{bmatrix}#1\\#2\\#3\\#4\end{bmatrix}}

%\pagestyle{fancy} 
\lfoot{Linear}
\rfoot{M1 study}
\renewcommand{\familydefault}{\sfdefault}



\begin{document}

\vspace*{-0.5in}
\begin{center}
 \textbf{ \large \sc{Midterm I Study Ideas} }
\end{center}

\textbf{Basic Info} You will have 1 hour to complete the midterm. It will cover chapters 1-3,5-6 with an emphasis on chapters 1-5. Everything from the homework and the quizzes is fair game. You can bring in a single page of notes with writing on the front and a calculator. Note that you will not need the calculator and are expected to show your work.

\textbf{Review by Topic} Here is a more specific list of things to know and more study ideas.

	\begin{itemize}
	\item Ch 1 
		\begin{itemize} 
		\item Know the language and notation of vectors and how to perform basic vector operations including rules of vector algebra.
		\item Know the terminology of \emph{linear combinations of vectors}, \emph{inner product of vectors}, and \emph{unit vectors}.
		\item Recall familiar applications.
		\end{itemize}
	\item Ch 2 
		\begin{itemize}
		\item Know the definition of a linear function and the basic strategies for determining if a function is or is not linear. Specifically, be able to show that a function is linear by exhibiting it in the form of
ab inner product: $f(x) = a^T x$ and be able to show that a function is not linear by finding a counter-example.
		\item Know the definition of an affine function and the basic strategy for determining if a function is or is not affine.
		\item Review familiar applications and interpretations of linear/affine functions.
		\end{itemize}
	\item Ch 3
		\begin{itemize}
		\item Know how to compute the norm of a vector, the distance between two vectors, and the angle between two vectors.
		\item Know how to tell if two vectors are orthogonal (perpendicular) or at an acute (or obtuse) angle.
		\item Know the triangle inequality and the Cauchy-Schwartz inequality.
		\item Be able to do algebraic manipulations involving norms and inner products
		\end{itemize}
	\item Ch 5
		\begin{itemize}
		\item Know the definition of linear independence and linear dependence and how to use them to \emph{show} that a set of vectors is or is not linearly independent (dependent).
		\item Know the \textbf{independence-dimension inequality} or \textbf{Fact A}.
		\item Know the definition of a basis and an orthonormal basis. Understand how to determine if a set of vectors is a basis and how to write one vector as a linear combination of others. What are the advantages of an orthonormal set of vectors?
		\item Know the Gram-Schmidt algorithm.  Be able to implement it in a simple case
(turn $a_1$, $a_2$, $a_3$ into $q_1$, $q_2$, $q_3$ with intermediate vectors
$\tilde q_k$.)  What properties do the $q$'s have? How are those properties related to the $a$'s?
		\item If $q_1$, $q_2$ and $q_3$ are orthonormal in $\Reals^3$, explain how you know they form a basis.  Since they do, given
$x\in\Reals^3$ we can write $x = \alpha_1 q_1+\alpha_2 q_2+ \alpha_3 q_3$
for some numbers $\alpha_1$, $\alpha_2$ and $\alpha_3$.  What are the numbers
$\alpha_k$?  Hint: they can be expressed using inner products: equation (5.5) in the text.
		\end{itemize}
	\item Ch 6
		\begin{itemize}
		\item Know how to reference matrices and to do basic calculations including matrix-vector multiplication.
		\end{itemize}


\end{itemize}

\end{document}
