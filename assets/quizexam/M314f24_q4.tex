\documentclass[11pt,fleqn]{article} 
\usepackage[margin=0.8in, head=0.8in]{geometry} 
\usepackage{amsmath, amssymb, amsthm}
\usepackage{fancyhdr} 
\usepackage{palatino, url, multicol}
\usepackage{graphicx} 
\usepackage[all]{xy}
\usepackage{polynom} 
\usepackage{pdfsync}
\usepackage{enumerate}
\usepackage{framed}
\usepackage{setspace}
\usepackage{array,tikz}

\newcommand{\bbm}{\begin{bmatrix}}
\newcommand{\ebm}{\end{bmatrix}}

\newcommand{\Reals}{\mathbb{R}}
\def\vectwo#1#2{\begin{bmatrix}#1\\#2\end{bmatrix}}
\def\vecthree#1#2#3{\begin{bmatrix}#1\\#2\\#3\end{bmatrix}}
\def\vecfour#1#2#3#4{\begin{bmatrix}#1\\#2\\#3\\#4\end{bmatrix}}

\pagestyle{fancy} 
\lhead{Linear Algebra}
\chead{Quiz 4}
\rhead{}

\begin{document}
\renewcommand{\headrulewidth}{0pt}
\newcommand{\blank}[1]{\rule{#1}{0.75pt}}
\renewcommand{\d}{\displaystyle}
%\vspace*{-0.7in}
This quiz is worth 10 points. \hfill {\Large{Name: \underline{\hspace{2in}}}}
\begin{enumerate}
\item (2 points) Very briefly, explain how you can conclude \emph{without any computation} that the vectors below are linearly \textbf{dependent}. \\


 $v_1=\vecfour 1 2 3 4, v_2=\vecfour 2 1 4 3, v_3=\vecfour1 {-1} 1 0,  v_4=\vecfour 1 0 1 0, v_5=\vecfour {\pi} {\sqrt{2}} 0 1$
 
\vspace{.8in}

\item (5 points) Show the vectors $a_1=\vecfour 1 0 0 1, a_2=\vecfour 1 1 1 0, a_3=\vecfour 0 {-1} 1 1$ are linearly \textbf{independent.}

 
\newpage
\item (3 points) Show that the vectors $x_1=\vecthree 1 1 0, x_2=\vecthree 1 0 1, x_3= \vecthree 0 1 {-1}$ are linearly \textbf{dependent}. \\

(Note: You don't need to look too hard.)

\vfill
\item (1 point bonus) Answer Question 3 above in a different way. 
\vfill
\end{enumerate}
\end{document}