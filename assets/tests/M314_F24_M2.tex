\documentclass[11pt,fleqn]{article} 
\usepackage[margin=0.8in, head=0.8in]{geometry} 
\usepackage{amsmath, amssymb, amsthm}
\usepackage{fancyhdr} 
\usepackage{palatino, url, multicol}
\usepackage{graphicx,tabularx,systeme} 
\usepackage[all]{xy}
\usepackage{polynom} 
\usepackage{pdfsync}
\usepackage{enumerate}
\usepackage{framed}
\usepackage{setspace}
\usepackage{array,tikz}

\newcommand{\bbm}{\begin{bmatrix}}
\newcommand{\ebm}{\end{bmatrix}}

\newcommand{\Reals}{\mathbb{R}}
\def\vectwo#1#2{\begin{bmatrix}#1\\#2\end{bmatrix}}
\def\vecthree#1#2#3{\begin{bmatrix}#1\\#2\\#3\end{bmatrix}}
\def\vecfour#1#2#3#4{\begin{bmatrix}#1\\#2\\#3\\#4\end{bmatrix}}

%\pagestyle{fancy} 
\lfoot{Linear}
\rfoot{M2}
\renewcommand{\familydefault}{\sfdefault}
\begin{document}
{\emph{\fontsize{26}{28}\selectfont Fall 2024 \hfill
%{\fontsize{32}{36}\selectfont Calculus 1: Midterm 1}
\hfill Math F314X}}

\begin{center}
{\emph{%\fontsize{26}{28}\selectfont Spring 2024 
%%\hfill
{\fontsize{32}{36}\selectfont Linear Algebra: Midterm 2}
%%\hfill Math F251X}
}}
\end{center}

\vfill

\quad \hfill {\Large{Name: \underline{\hspace{2in}}}}

\vfill

{\fontsize{18}{22}\selectfont\emph{Rules:}}

\begin{itemize}
\item Show your work.

\item You may have a single handwritten sheet of paper with writing on one side.

\item You may use a calculator 

\end{itemize}


\vfill

\def\emptybox{\hbox to 2em{\vrule height 16pt depth 8pt width 0pt\hfil}}
\def\tline{\noalign{\hrule}}
\centerline{\vbox{\offinterlineskip
{
\bf\sf\fontsize{18pt}{22pt}\selectfont
\hrule
\halign{
\vrule#&\strut\quad\hfil#\hfil\quad&\vrule#&\quad\hfil#\hfil\quad
&\vrule#&\quad\hfil#\hfil\quad&\vrule#\cr
height 3pt&\omit&&\omit&&\omit&\cr
&Problem&&Possible&&Score&\cr\tline
height 3pt&\omit&&\omit&&\omit&\cr
&1&&20&&\emptybox&\cr\tline
&2&&20&&\emptybox&\cr\tline
&3&&20&&\emptybox&\cr\tline
&4&&20&&\emptybox&\cr\tline
&5&&20&&\emptybox&\cr\tline
&Total&&100&&\emptybox&\cr
}\hrule}}}

\vfill

\newpage

\quad

\newpage
\begin{enumerate}

%convolution
\item (20 points) Let $a=(1,-1)$ and $x=(x_1,x_2, \cdots, x_{20}).$ Let $b=a*x,$ the convolution of the 2-vector $a$ and the 20-vector $x$.
	\begin{enumerate}
	\item What are the dimensions of $b$?
	\vspace{1in}
	\item Find the entries of $b$. (Note that $b$ is too long to list all of the entries. It might be useful to let $b=(b_1,b_2, \cdots, b_n)$ and describe $b_i$ for appropriate $i$-values.)
	\vfill
	\item We know there is a matrix $A$ such that $Ax=a*x.$
		\begin{enumerate}
		\item What are the dimensions of the matrix $A$.
		\vspace{1in}
		\item Describe the first three columns of $A.$
		\vfill
		\end{enumerate}
	\end{enumerate}
\newpage
%matrix product and LR inverses
\item (20 points) The 3-vector $c$ represents the coefficients of the quadratic polynomial $p(x)=c_1+c_2x+c_3x^2.$ 
	\begin{enumerate}
	\item Find a matrix $B$ such that $Bc=\int_0^2 p(x) \: dx.$
	\vfill
	\item Express the conditions $p(0)=0,$ $p'(0)=1$, $p(2)=-3$, and $p'(2)=0$ as a set of linear equations of the form $Ac=b.$
	\vfill
	\item Does $A$ have a left inverse? Justify your answer. (Note: You are not asked to find a left inverse, only determine whether or not one exists.)
	\vspace{1in}
	\item Does $A$ have a right inverse? Justify your answer. 
	\vspace{1in}
	\end{enumerate}
\newpage
%QR factorization
\item (20 points) Suppose $A=\begin{bmatrix}
1&1&0\\0&1&2\\0&-1&1 \end{bmatrix}.$ 
	\begin{enumerate}
	\item Suppose someone, say Jill, gives you a matrix $Q$ and a matrix $R$ and asserts they are a QR-factorization of $A$. What do you need to check to confirm Jill is correct? Be specific and be efficient.
	\vfill
	\item In fact, $A$ does have QR-factorization where $Q=\begin{bmatrix} 1&0&0\\ \star& \sqrt{2}/2&\sqrt{2}/2\\ \star& -\sqrt{2}/2&\sqrt{2}/2\end{bmatrix}$ and $R=\begin{bmatrix} 1&1&\star \\ 0& \sqrt{2}&\star \\ 0& \star&\star \end{bmatrix}.$\\
	
	Fill in the blank spaces in the entries of $Q$ and $R$ below. You can show your work and/or explain your reasoning in the remaining space.
	
	
	Answer: \\
	
	$Q=\begin{bmatrix} 1&0&0\\ \framebox(30,30){}& \sqrt{2}/2&\sqrt{2}/2\\ \framebox(30,30){}& -\sqrt{2}/2&\sqrt{2}/2\end{bmatrix}$ and $R=\begin{bmatrix} 1&1&\framebox(30,30){} \\ 0& \sqrt{2}&\framebox(30,30){} \\ 0& \framebox(30,30){}&\framebox(30,30){} \end{bmatrix}.$\\
	
	%\item Use the $Q$ and $R$ from part (b) to solve the system $Ax=\vecthree{1}{0}{2\sqrt{2}}.$
	\vspace{3in}
	\end{enumerate}
\newpage
\item (20 points) Let $A=\begin{bmatrix} 2&2&-1\\0&-2&1\\0&0&1\end{bmatrix}.$
	\begin{enumerate}
	\item Find a left inverse of $A.$ (You can do this by inspection.)
	\vfill
	\item Does $A$ have a right inverse? Justify your answer. (You are not asked to find one, just to determine whether or not one exists.)
	\vspace{1in}
	\item Let $b=(4,-6,3).$
		\begin{enumerate} 
		\item Use your answer in part (a) to solve the system of equations $Ax=b.$
		\vfill
		\item Since $A$ happens to be upper triangular, describe an alternative method for solving $Ax=b.$ (You don't have to perform the method, though you could use it to check your answer.)
		\vspace{1in}
		\end{enumerate}
		
	\end{enumerate}
\newpage
\item (20 points) 
	\begin{enumerate}
	\item Give an example of a 3 by 3 matrix $A$ such that all entries are non-zero, all entries of $A$ are distinct (no repeats) and such that $A$ is \textbf{NOT} invertible. Explain why your example is not invertible.
	\vfill
	\item Assume $A$ is square, has linearly independent columns, and has a $QR$ factorization. Write its inverse $A^{-1}$ in terms of $Q^T$ and $R^{-1}$. 	\vfill
	\item Suppose $B$ is an orthogonal matrix.
		\begin{enumerate}
		\item State the \textbf{definition} of an orthogonal matrix.
		\vspace{1in}
		\item Are the rows of $B$ linearly independent? Justify your answer.
		\vfill
		\end{enumerate}
	\item Suppose $A$ is an $m \times n$ matrix. Suppose $D$ is an $m \times m$ diagonal matrix such that $D_{ii}=2^i.$ Describe the result of matrix multiplication $DA.$
	\vfill
	\end{enumerate} 
\end{enumerate}
\end{document}

%%%%ENDDOCUMENT


