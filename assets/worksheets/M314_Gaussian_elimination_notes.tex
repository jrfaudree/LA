\documentclass[11pt,fleqn]{article} 
\usepackage[margin=0.8in, head=0.8in]{geometry} 
\usepackage{amsmath, amssymb, amsthm}
\usepackage{fancyhdr} 
\usepackage{palatino, url, multicol}
\usepackage{graphicx,tabularx,systeme} 
\usepackage[all]{xy}
\usepackage{polynom} 
\usepackage{pdfsync}
\usepackage{enumerate}
\usepackage{framed}
\usepackage{setspace}
\usepackage{array,tikz}

\newcommand{\bbm}{\begin{bmatrix}}
\newcommand{\ebm}{\end{bmatrix}}

\newcommand{\Reals}{\mathbb{R}}
\def\vectwo#1#2{\begin{bmatrix}#1\\#2\end{bmatrix}}
\def\vecthree#1#2#3{\begin{bmatrix}#1\\#2\\#3\end{bmatrix}}
\def\vecfour#1#2#3#4{\begin{bmatrix}#1\\#2\\#3\\#4\end{bmatrix}}

\pagestyle{fancy} 
\lfoot{Linear}
\rfoot{Ch 6}
\renewcommand{\familydefault}{\sfdefault}
\begin{document}

\renewcommand{\headrulewidth}{0pt}
\newcommand{\blank}[1]{\rule{#1}{0.75pt}}
\renewcommand{\d}{\displaystyle}
\vspace*{-0.7in}
\begin{center}
 \textbf{ \large \sc{Notes: Solutions to Systems of Linear Equations and Gaussian Elimination} }
\end{center}

\begin{enumerate}
\item Systems of Linear Equations\\

\begin{tabularx}{\textwidth}{XXX}
$S_1=\systeme{x+y=4,x+2y=-1}$&$S_2=\systeme{x+2y=1,x+2y=2}$&$S_3=\systeme{x_1+x_2+3x_3=5,x_1+2x_2+4x_3=6}$\\
&&\\
&&\\
\textbf{Encoding}&&\\
&Matrix-vector Multiplication&\\
&&\\
$\bbm 1&1\\1&2 \ebm \vectwo x y = \vectwo 4 {-1}$&$\bbm 1&2\\1&2 \ebm \vectwo x y = \vectwo 1 {2}$&
$\bbm 1&1&3\\1&2&4 \ebm \vecthree {x_1}{x_2}{x_3} y = \vectwo 5 {6}$ \\
&&\\
&Augmented Matrix&\\
&&\\
$\bbm 1&1&4\\1&2&-1 \ebm$ & $\bbm 1&2&1\\1&2&2 \ebm$ & $\bbm 1&1&3&5\\1&2&4&6 \ebm$ \\
\end{tabularx}

\vspace{.2in}

\item A \textbf{solution} to a system of equations is a set of values (numbers) for the variables so that \emph{all} equations in the system are true.\\

\begin{tabularx}{\textwidth}{XXX}
$S_1=\systeme{x+y=4,x+2y=-1}$&$S_2=\systeme{x+2y=1,x+2y=2}$&$S_3=\systeme{x_1+x_2+3x_3=5,x_1+2x_2+4x_3=6}$\\
&&\\
&&\\
\textbf{A solution}&&\\
&&\\
$x=9,\: y=-5$& none & $x_1=2,x_2=0,x_3=1$\\
& & $x_1=4,x_2=1,x_3=0$\\
\end{tabularx}

\item \textbf{To solve} as system of equations is to find \emph{all possible solutions}.
	\begin{itemize}
	\item For system $S_1:$ $x=9,\: y=-5$ is the only solution.\\
	\item For system  $S_2:$ no solutions\\
	\item For system $S_3:$ an infinite number of solutions. Specifically, the solution set can be written\\
	
	$$ \left\{ \vecthree{4-2x_3}{1-x_3}{x_3} \:  : \: x_3 \text{ is any real number } \right\}$$
	
	\end{itemize}
\newpage
\item \textbf{Observation:} The operations below do not change the solutions to a system of equations.\\

\begin{tabularx}{\textwidth}{XXX}
(a) Reordering the equations & $S_1=\systeme{x+y=4,x+2y=-1}$&$S'_1=\systeme{x+2y=-1,x+y=4}$\\
&&\\
(b) Multiplying an equation by a constant&$S_1=\systeme{x+y=4,x+2y=-1}$&$S'_1=\systeme{\pi x+\pi y=4\pi,x+2y=-1}$\\
 &&\\
 (c) Adding a multiple of one equation to another &$S_1=\systeme{x+y=4,x+2y=-1}$&$S'_1=\systeme{x+y=4,3x+4y=7}$\\

\end{tabularx}

\vspace{.2in}

NOTE: The equation \fbox{$3x+4y=7$} is obtained by \fbox{ (equ. 2) $+$ 2(equ. 1)} or, equivalently,  \\
\fbox{ $(x+2y)+2(x+y)=-1+2(4).$}

\vspace{.2in}


\item \textbf{Observation:} To operations in item \#4 above can be described by \textbf{row operations} performed on the augmented matrix.\\

These are called \textbf{elementary row operations}.\\

Encode  \: \fbox{$S_1=\systeme{x+y=4,x+2y=-1}$} \: as  \: $A=\bbm 1&1&4\\1&2&-1 \ebm$\\

\vspace{.2in}


\begin{tabularx}{\textwidth}{XXXX}
(a) Reorder rows& $\bbm 1&1&4\\1&2&-1 \ebm$&$r_1 \leftrightarrow r_2$&$\bbm 1&2&-1\\1&1&4\\ \ebm$\\
&&\\
(b) Multiply a row by a constant& $\bbm 1&1&4\\1&2&-1 \ebm$&$\pi*r_1 \to r_1$&$\bbm \pi&\pi&4\pi\\1&2&-1 \ebm$\\
 &&\\
 (c) Adding a multiple of one row to another &$\bbm 1&1&4\\1&2&-1 \ebm$&$2r_1+r_2 \to r_2$&$\bbm 1&1&4\\3&4&7 \ebm$\\
\end{tabularx}

\item Gaussian Elimination\\

Solve a system of equations by \\

(1) encoding the system as an augmented matrix, \\

(2) repeatedly use elementary row operations to construct the matrix of an \textbf{equivalent} system that is \textbf{simple} to solve, then \\

(3) solve the new simple system.
	
\end{enumerate}
\end{document}