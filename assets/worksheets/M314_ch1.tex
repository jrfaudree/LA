\documentclass[11pt,fleqn]{article} 
\usepackage[margin=0.8in, head=0.8in]{geometry} 
\usepackage{amsmath, amssymb, amsthm}
\usepackage{fancyhdr} 
\usepackage{palatino, url, multicol}
\usepackage{graphicx} 
\usepackage[all]{xy}
\usepackage{polynom} 
\usepackage{pdfsync}
\usepackage{enumerate}
\usepackage{framed}
\usepackage{setspace}
\usepackage{array,tikz}

\newcommand{\bbm}{\begin{bmatrix}}
\newcommand{\ebm}{\end{bmatrix}}

\pagestyle{fancy} 
\lfoot{Linear}
\rfoot{Ch1}
\renewcommand{\familydefault}{\sfdefault}
\begin{document}

\renewcommand{\headrulewidth}{0pt}
\newcommand{\blank}[1]{\rule{#1}{0.75pt}}
\renewcommand{\d}{\displaystyle}
\vspace*{-0.7in}
\begin{center}
  \large \sc{Worksheet: Vector Operations}
\end{center}

Let $\d v=\begin{bmatrix} 1 \\ -2 \\ 3 \end{bmatrix},$ $w=\bbm 4\\1\\-1 \ebm,$  $u=\bbm 1\\4 \ebm,$ $z=(2,1).$
\begin{enumerate}
\item On the same set of axes, draw $u,$ $z$ and $u+z$.
\vfill
\item On the same set of axes, dray $u$, $z$ and $u-z$.
\vfill
\item Make the calculations below or explain why it is not defined.\\
	\begin{multicols}{2}
	\begin{enumerate}
	\item $v+u$\\
	
	\vspace{.5in}
	
	\item $2v+w$\\
		
\vspace{.5in}
	
	\item $5 \, \textbf{1}_4 - (u,u)$\\
		\vspace{.5in}	
	\item $vw$\\ \vspace{.5in}

	
	\item $v^Tw$\\
		\vspace{.5in}

	
	\item $w^Tv$\\
		\vspace{.5in}

	
	\item $(w^Tv)u$\\
		\vspace{.5in}

	
	\item $(w^Tv)+u$\\
\vspace{.5in}
	
	\item $((w^Tv),1)+u$\\
		\vspace{.5in}

	\end{enumerate}
	\end{multicols}
\vspace{.5in}



\newpage
\item Find $y_3$ and $y_{2:4}$ for $y=(2v,u).$ Recall $\d v=\begin{bmatrix} 1 \\ -2 \\ 3 \end{bmatrix}$ and  $u=\bbm 1\\4 \ebm.$ 
\vfill

\item Suppose $x$ is a vector of dimension 100 and $\textbf{1}=\textbf{1}_{100}$. Use words and symbols (such as $x_i$) to describe what each calculation below will do.
	\begin{enumerate}
	\item $\textbf{1}^Tx$\\
	\vfill
	\item $\left(\frac{\textbf{1}^T}{100}\right) x$
	\vfill
	\item $\sqrt{x^Tx}$
	\vfill
	\item $\left(e_1+e_2 \right)^Tx$
	\vfill
	\item Construct a vector $a$ such that $a^Tx$ gives the average of the last 10 entries in $x$.
	\vfill
	\end{enumerate}
\end{enumerate}
\end{document}