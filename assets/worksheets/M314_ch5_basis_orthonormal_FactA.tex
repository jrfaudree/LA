\documentclass[11pt,fleqn]{article} 
\usepackage[margin=0.8in, head=0.8in]{geometry} 
\usepackage{amsmath, amssymb, amsthm}
\usepackage{fancyhdr} 
\usepackage{palatino, url, multicol}
\usepackage{graphicx,tabularx} 
\usepackage[all]{xy}
\usepackage{polynom} 
\usepackage{pdfsync}
\usepackage{enumerate}
\usepackage{framed}
\usepackage{setspace}
\usepackage{array,tikz}

\newcommand{\bbm}{\begin{bmatrix}}
\newcommand{\ebm}{\end{bmatrix}}

\pagestyle{fancy} 
\lfoot{Linear}
\rfoot{Ch 5}
\renewcommand{\familydefault}{\sfdefault}
\begin{document}

\renewcommand{\headrulewidth}{0pt}
\newcommand{\blank}[1]{\rule{#1}{0.75pt}}
\renewcommand{\d}{\displaystyle}
\vspace*{-0.7in}
\begin{center}
  \large \sc{Worksheet: Fact A, Basis, and Orthonormal Vectors}
\end{center}

\begin{enumerate}
\item Fact A: \\
\vspace{1in}
\item Definition: A \emph{basis} is\\
\vspace{1.5in}
\item Give three distinct examples of bases when
	\begin{enumerate}
	\item $n=2$\\
	\vfill
	\item $n=3$\\
	\vfill
	\end{enumerate}
\item Fact B: 
\vfill
\newpage
\item A set of $n$-vectors $a_1,a_2,\cdots,a_k$ is called \emph{orthogonal} if
\vspace{1in}
\item Examples\\
\vfill
\item A vector $a$ is called \emph{normal} if 
\vspace{.5in}
\item Examples\\
\vfill
\item A set of $n$-vectors $a_1,a_2,\cdots,a_k$ is called \emph{orthonormal} if
\vspace{1.5in}
\item Examples\\
\vfill
\newpage
\item Suppose $a_1,a_2,a_3,$ and $a_4$ is a set of orthonormal $32$-vectors. Further, suppose that $\beta_1, \beta_2, \beta_3$ and $\beta_4$ have the property that 
$$\beta_1a_1+\beta_2 a_2 +\beta_3 a_3 + \beta_4 a_4=0_{32}.$$
	\begin{enumerate}
	\item Find $a_3^T(\beta_1a_1+\beta_2 a_2 +\beta_3 a_3 + \beta_4 a_4)$.
	\vfill
	\item Find $a_3^T0_{32}.$
	\vfill
	\item What can you conclude about $\beta_3$? About $\beta_i$ for $i=1,2,4$?
	\vfill
	\item What can you conclude about the set $a_1,a_2,a_3,$ and $a_4$? About \emph{any} set of orthonormal vectors?
	\vfill
	\end{enumerate}
\end{enumerate}
\end{document}