\documentclass[11pt,fleqn]{article} 
\usepackage[margin=0.8in, head=0.8in]{geometry} 
\usepackage{amsmath, amssymb, amsthm}
\usepackage{fancyhdr} 
\usepackage{palatino, url, multicol}
\usepackage{graphicx} 
\usepackage[all]{xy}
\usepackage{polynom} 
\usepackage{pdfsync}
\usepackage{enumerate}
\usepackage{framed}
\usepackage{setspace}
\usepackage{array,tikz}

\newcommand{\bbm}{\begin{bmatrix}}
\newcommand{\ebm}{\end{bmatrix}}

\pagestyle{fancy} 
\lfoot{Linear}
\rfoot{Ch 2}
\renewcommand{\familydefault}{\sfdefault}
\begin{document}

\renewcommand{\headrulewidth}{0pt}
\newcommand{\blank}[1]{\rule{#1}{0.75pt}}
\renewcommand{\d}{\displaystyle}
\vspace*{-0.7in}
\begin{center}
  \large \sc{Worksheet: Vector Algebra, Linear and Affine Functions}
\end{center}
\begin{enumerate}
\item  Label each of the statements below TRUE or FALSE.\\

Let $a, u,$ and $v$ be $n$-vectors and let $\alpha$ and $\beta$ be scalars.

\begin{multicols}{3}
\begin{enumerate}
\item $a^Tu=u^Ta$\\

\vspace{.3in}

\item $\alpha(u+v)=\alpha u+ \alpha v$\\

\columnbreak

\item $\alpha(a^Tu)=(\alpha a)^Tu$\\

\vspace{.3in}

\item $\alpha(a^Tu)=a^T(\alpha u)$\\

\columnbreak

\item $a^T(u+v)=a^Tu+a^Tv$\\

\vspace{.3in}

\item $\beta(a^Tu)+\beta=\beta(a^Tu+1)$\\
\end{enumerate}
\end{multicols}
\vspace{.3in}
\item Complete the definition of a \emph{linear vector function}:\\
The function $f: \mathbb{R}^n \to \mathbb{R}$ is linear if for every pair of vectors $u$ and $v$ and every pair of scalars $\alpha$ and $\beta,$

\vspace{.3in}

\item Make up two examples of functions $f:\mathbb{R}^3 \to \mathbb{R}$, one that is linear and one that is not linear.

\vfill

\item Every linear function can be written

\vspace{1in}

\item The definition of an \emph{affine vector function}:\\
The function $f: \mathbb{R}^n \to \mathbb{R}$ is \emph{affine} if for every pair of vectors $u$ and $v$ and every pair of scalars $\alpha$ and $\beta$ 

\vspace{1in}

\item Every affine function can be written

\vspace{1in}

\newpage
\item Linear Taylor Approximations

\vfill

\item Let $f(x)=x_1e^{-x_2}+x_3$ and $z=(2,0,1).$
	\begin{enumerate}
	\item Find $\widehat{f}(x)$, the linear Taylor approximation of $f$ at $z.$
	\vfill
	\item Find ${f}(2.1,0.1,0.9)$ and $\widehat{f}(2.1,0.1,0.9).$
	\vfill
	\end{enumerate}
\newpage
\item Linear Regression
\vfill
\item In the chart below, $x_1$ is house area in 1000 square feet and $x_2$ is the number of bedrooms. Assume the coefficient  vector is $\beta=(148.73,-18.85)$ and $v=54.40$.\\
\includegraphics{book_house_chart}

Write out the linear approximation $\widehat{y}$ given by $\beta$ and $v$ and confirm that the top entry in the last column is correct.

\vfill

\item Interpret the coefficients in $\beta$.
\vfill

\end{enumerate}
\end{document}