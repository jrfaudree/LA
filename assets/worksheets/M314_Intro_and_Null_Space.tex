\documentclass[11pt,fleqn]{article} 
\usepackage[margin=0.8in, head=0.8in]{geometry} 
\usepackage{amsmath, amssymb, amsthm}
\usepackage{fancyhdr} 
\usepackage{palatino, url, multicol}
\usepackage{graphicx,tabularx,systeme} 
\usepackage[all]{xy}
\usepackage{polynom} 
\usepackage{pdfsync}
\usepackage{enumerate}
\usepackage{framed}
\usepackage{setspace}
\usepackage{array,tikz}

\newcommand{\bbm}{\begin{bmatrix}}
\newcommand{\ebm}{\end{bmatrix}}

\newcommand{\Reals}{\mathbb{R}}
\def\vectwo#1#2{\begin{bmatrix}#1\\#2\end{bmatrix}}
\def\vecthree#1#2#3{\begin{bmatrix}#1\\#2\\#3\end{bmatrix}}
\def\vecfour#1#2#3#4{\begin{bmatrix}#1\\#2\\#3\\#4\end{bmatrix}}

\pagestyle{fancy} 
\lfoot{Linear}
\rfoot{Intro + Null Spaces}
\renewcommand{\familydefault}{\sfdefault}
\begin{document}

\renewcommand{\headrulewidth}{0pt}
\newcommand{\blank}[1]{\rule{#1}{0.75pt}}
\renewcommand{\d}{\displaystyle}
\vspace*{-0.7in}
\begin{center}
 \textbf{ \large \sc{Motivation for Eigenvalues and Eigenvectors} }
\end{center}
\begin{enumerate}
\item An example of a system of linear, first-order differential equations.\\

\begin{tabular}{lll}
$\frac{dv}{dt}=v(t)-w(t)$&\quad&$v(0)=40$\\
&&\\
$\frac{dw}{dt}=2v(t)-4w(t)$&\quad&$w(0)=10$
\end{tabular}

\item A solution: $v(t)=90e^{2t}-50e^{3t}, \quad w(t)=-90e^{2t}+100e^{3t}.$
\newpage

\begin{center} Null Spaces \end{center}
\item Let $A$ be an $m \times n$ matrix, then the \textbf{null space of $A$} is
\vspace{1in}
\item Example 1: $A=\bbm 1&2\\10&20 \ebm$\\

\vfill

\item Example 2: $B=\bbm 1&0&1 \\ 5&4&9\\ 2&4&6 \ebm$\\
\vfill
\newpage
\item Show/prove/give an argument for each of the statements below.\\
	\begin{enumerate}
	\item For every matrix $A$, $N(A) \not = \emptyset.$
	\vspace{.5in}
	\item If $A$ is invertible, then $N(A)$ contains only the zero vector.\\
	\vspace{.5in}
	\item If the vector $x$ is in $N(A),$ then for any number $c$, $cx$ is in $N(A)$.\\
	\vfill
	\item If both of the vectors $x$ and $y$ are in $N(A),$ then the vector $x+y$ is also in $N(A).$\\
	\vfill
	\item If $x$ is in $N(A)$ and the vector $z$ is \emph{not} in $N(A)$, then $x$ and $z$ are linearly independent.
	\vfill
	\item If the vector $a$ is in $N(A)$ and $c$ is a solution to $Ax=b,$ then $c+a$ is also a solution to $Ax=b.$
	\vfill
	\item If both of the vectors $c_1$ and $c_2$ are solutions to $Ax=b,$ then there is a vector $a$ in $N(A)$ such that $c_2=c_1+a.$
	\vfill
	\end{enumerate}
\newpage
\item Main Principles
\end{enumerate}
\end{document}
	