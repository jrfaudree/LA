\documentclass[11pt,fleqn]{article} 
\usepackage[margin=0.8in, head=0.8in]{geometry} 
\usepackage{amsmath, amssymb, amsthm}
\usepackage{fancyhdr} 
\usepackage{palatino, url, multicol}
\usepackage{graphicx,tabularx,systeme} 
\usepackage[all]{xy}
\usepackage{polynom} 
\usepackage{pdfsync}
\usepackage{enumerate}
\usepackage{framed}
\usepackage{setspace}
\usepackage{array,tikz}

\newcommand{\bbm}{\begin{bmatrix}}
\newcommand{\ebm}{\end{bmatrix}}

\newcommand{\Reals}{\mathbb{R}}
\def\vectwo#1#2{\begin{bmatrix}#1\\#2\end{bmatrix}}
\def\vecthree#1#2#3{\begin{bmatrix}#1\\#2\\#3\end{bmatrix}}
\def\vecfour#1#2#3#4{\begin{bmatrix}#1\\#2\\#3\\#4\end{bmatrix}}

\pagestyle{fancy} 
\lfoot{Linear}
\rfoot{Ch 6}
\renewcommand{\familydefault}{\sfdefault}
\begin{document}

\renewcommand{\headrulewidth}{0pt}
\newcommand{\blank}[1]{\rule{#1}{0.75pt}}
\renewcommand{\d}{\displaystyle}
\vspace*{-0.7in}
\begin{center}
 \textbf{ \large \sc{Notes: Reduced Row-Echelon Form and Gaussian Elimination with Partial Pivoting} }
\end{center}

\begin{enumerate}
\item Examples from Monday\\

\begin{tabularx}{0.9\textwidth}{XXX}
$S_1=\systeme{x+y=1, 2y-z=-4,x+y+z=4}$ & $\bbm 1&1&0&5\\0&2&-1&-4 \\ 1&1&1&4 \ebm$ & $\bbm 1&0&0&3/2\\0&1&0&-1/2\\0&0&1&3 \ebm$ \\

&&\\

$S_3=\systeme{x_1+x_2+3x_3=5,x_1+2x_2+4x_3=6}$ & $\bbm 1&1&3&5\\1&2&4&6 \ebm$ &$\bbm 1&0&2&4 \\ 0&1&1&1 \ebm$ \\

\end{tabularx}

\vfill

\item Reduced Row Echelon Form
	\begin{enumerate}
	\item Rows of all zeros are at the bottom.
	\item Every row with nonzero entries has a 1 in the left-most entry. (Called the \textbf{leading one} or \textbf{pivot})
	\item If a row has a leading 1, it is to the right of all leading 1's in the rows above.
	\item Each column with a leading 1 has zeros in all other entries.
	\end{enumerate}
	
\vfill

\item Example A\\

\begin{tabularx}{\textwidth}{XX}
$S_4=\systeme{v+2w+y=-1,2v+4w+x+y=0, -v-2w+x-2y+2z=11,v+2w+x+z=5}$ & $\bbm 1&2&0&1&0&-1\\ 2&4&1&1&0&0\\-1&-2&1&-2&2&11\\1&2&1&0&1&5 \ebm$\\
&\\
 &$\bbm 1&2&0&1&0&-1\\0&0&1&-1&0&2\\0&0&0&0&1&4\\0&0&0&0&0&0 \ebm$ \\
\end{tabularx}

\vspace{2in}
\newpage
\item Example B: Solve $\systeme{2w+4x-2y-2z=-4,w+2x+4y-3z=5,-3w-3x+8y-2z=7,-w+x+6y-3z=7}.$\\

\vfill
\begin{tabularx}{\textwidth}{XXXX}
$\bbm 2&4&-2&-2&-4\\1&2&4&-3&5\\-3&-3&8&-2&7\\-1&1&6&-3&7 \ebm$ & $r_2-(1/2)r_1 \to r_2$&$r_3+(3/2)r_1 \to r_3$&$r_4+(1/2)r_1 \to r_4$\\
\end{tabularx}

\vfill

\begin{tabularx}{\textwidth}{XXXX}
$\bbm 2&4&-2&-2&-4\\0&0&5&-2&7\\0&3&5&-5&1\\0&3&5&-4&5 \ebm$ &$r_2 \leftrightarrow r_4$&&\\
\end{tabularx}

\vfill

\begin{tabularx}{\textwidth}{XXXX}$\bbm 2&4&-2&-2&-4\\0&3&5&-4&5\\0&3&5&-5&1\\ 0&0&5&-2&7\ebm$ &$r_3-r_2 \to r_3$&&\\
\end{tabularx}

\vfill

\begin{tabularx}{\textwidth}{XXXX}$\bbm 2&4&-2&-2&-4\\0&3&5&-4&5\\0&0&0&-1&-4\\ 0&0&5&-2&7\ebm$ &$r_4 \leftrightarrow r_3$&&\\
\end{tabularx}

\vfill

\begin{tabularx}{\textwidth}{XXXX}$\bbm 2&4&-2&-2&-4\\0&3&5&-4&5\\ 0&0&5&-2&7\\0&0&0&-1&-4\ebm$ &keep going&&\\

\end{tabularx}
\vfill




\end{enumerate}
\end{document}