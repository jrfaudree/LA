% !TEX TS-program = pdflatexmk
\documentclass[12pt]{article}

% Layout.
\usepackage[top=.75in, bottom=0.75in, left=.75in, right=.75in, headheight=1in, headsep=6pt]{geometry}

% Fonts.
\usepackage{mathptmx}
%\usepackage[scaled=0.86]{helvet}
\renewcommand{\emph}[1]{\textsf{\textbf{#1}}}
\renewcommand{\familydefault}{\sfdefault}
%Syl adendum
\usepackage[colorlinks = true,linkcolor = blue, urlcolor  = blue]{hyperref}
\def\mailto#1{\href{mailto:#1}{#1}}

% Misc packages.
\usepackage{amsmath,amssymb,latexsym,multicol}
\usepackage{graphicx}
\usepackage{array}
\usepackage{xcolor}
\usepackage{multicol}
\usepackage{tabularx,colortbl}
\usepackage{enumitem}
%to make tikz pics work
\usepackage{tikz,pgfplots}

\usepackage[colorlinks=true]{hyperref}

% Paragraph spacing
\parindent 0pt
\parskip 6pt plus 1pt
\def\tableindent{\hskip 0.5 in}
\def\ts{\hskip 1.5 em}

\usepackage{fancyhdr}
\pagestyle{fancy} 
\lhead{\large\sf\textbf{MATH 314 }}
\rhead{\large\sf\textbf{Fall 2024}}
\chead{\large\sf\textbf{Linear Algebra}}

\newcommand{\localhead}[1]{\par\smallskip\textbf{#1}\nobreak\\}%
\def\heading#1{\localhead{\large\emph{#1}}}
\def\subheading#1{\localhead{\emph{#1}}}

\newenvironment{clist}%
{\bgroup\parskip 0pt\begin{list}{$\bullet$}{\partopsep 4pt\topsep 0pt\itemsep -2pt}}%
{\end{list}\egroup}%

\usetikzlibrary{calc}
\pgfplotsset{my style/.append style={axis x line=middle, axis y line=
middle, xlabel={$x$}, ylabel={$y$}, axis equal }}


\begin{document}

\textbf{\large{Essential Information}}

\begin{tabular}{p{0.2\textwidth} p{0.7\textwidth}}
{time \& place}:&MWF 9:15am-10:15am Chapman Hall Room 104\\
{instructor:} &Jill Faudree\\
{contact details:} &Chapman 306B, jrfaudree@alaska.edu, 474-7385\\
{office hours:} &(\textbf{tentative})  MWF 10:30am-11:30am \\
&and by appointment or just drop by.\\
&Current schedule available \href{https://docs.google.com/spreadsheets/d/1o710N0-GCNc8ICkVuf-IIC80N04p9w9wGnPADjhTMv0/pubhtml}{\texttt{online}}\\
{textbook:}& \textbf{Introduction to Applied Linear Algebra -- Vectors, Matrices, and Least Squares}, Stephen Boyd \& Lieven Vanderberghe\\
{webpage:}& \url{https://jrfaudree.github.io/LinearAlgebra2024/}\\
canvas:&\url{https://canvas.alaska.edu/courses/21607}\\
{prerequisites:} & MATH 252 Calculus 2 or permission of instructor.
\end{tabular}



\textbf{\large{Course Description}}

The course description in the catalog reads as follows:
\begin{quote} Linear equations, finite dimensional vector spaces, matrices, determinants, linear transformations and characteristic values. Inner product spaces. \end{quote}


\textbf{\large{Student Learning Outcomes}} A successful student in this course will:
\begin{itemize}
\item become fluent in linear combinations, bases, and matrix algebra
\item apply linear algebra tools to applied problems such as dynamical systems and least squares fitting
\item understand how QR factorization is used to solve matrix equations, including non-square systems
\item use software to solve practical problems using skills from linear algebra
\end{itemize}

{\textbf{\large{Course Mechanics}}}

Class meetings will be a combination of lecture, whole class discussion, individual work, and small group work. Always bring paper and a pen or pencil. 

{\textbf{\large{Homework}}}

There will be a homework assignment due roughly every week, usually on Wednesdays. Each week’s assignment and due date will be announced in class and will be posted on the public webpage. Answers to homework problems will be posted in advance, and homework will be graded solely on completion.

Regarding late homework, I will accept from each student a single late homework with no
questions asked. Simply hand in a note indicating you are using your free late homework in
place of your actual assignment. You must notify me no later than the time the homework is
due that you intend to take advantage of this opportunity, and you must hand in the homework no later than one week after it was due. Subsequent late homework will be accepted only under extenuating circumstances.

The late homework freebie cannot be used for the first two homework sets, nor can it be used for the final assignment.

{\textbf{\large{Julia}}}

Real-world problems in linear algebra are solved using computer software. We will use the software program Julie (a kind of modern Mathlab) on assignments and for the labs. No previous programming experience is required. I will provide materials and background sufficient to guide you through all Julia assignments.

{\textbf{\large{Quizzes}}}

Every Friday there will be a 15 minute quiz based on the homework handed in the previous Wednesday.

%{\textbf{\large{Labs}}}
%
%In addition to the more routine homework, there will be about four labs (i.e short projects) covering more in-depth material. More details on the labs will be announced along with your first lab.

{\textbf{\large{Midterms}}}

There will be two in-class midterms.

{\textbf{\large{Final Exam}}}

There will be a comprehensive Final Exam 8:00am-10:00am on Wednesday December 11.

{\textbf{\large{Rubric}}}

\begin{multicols}{2}
\textbf{Grades} will be calculated according to the following rubric:


\begin{tabular}{|l|c|}
  \hline
  % after \\: \hline or \cline{col1-col2} \cline{col3-col4} ...
  homework & 15\% \\
  quizzes & 20\%\\
  midterm 1 & 20\%\\
  midterm 2 & 20\%\\
  final exam & 25\% \\
  \hline
\end{tabular}
\end{multicols}

\begin{multicols}{2}
Letter grades will be assigned according to the following scale. This scale is a guarantee; I reserve the right to lower the thresholds.

\begin{tabular}{| llllllll |}
\hline
A+&97-100\%& \hspace{.1in}&C+&77-79\%& \hspace{.1in}&F&$\leq$ 59\\
A&93-96\%&&C&73-76\%&&&\\
A-&90-92\%&&C-&70-72\%&&&\\
B+&87-89\%& \hspace{.1in}&D+&67-69\%&&&\\
B&83-86\%&&D&63-66\%&&&\\
B-&80-82\%&&D-&60-62\%&&&\\
\hline
\end{tabular} 

\end{multicols}

\heading{Office Hours and Communication}
My Weekly Schedule including office hours are available and updated \href{https://docs.google.com/spreadsheets/d/1o710N0-GCNc8ICkVuf-IIC80N04p9w9wGnPADjhTMv0/pubhtml}{\texttt{online}}.  Students can also schedule meetings with me outside of regular office hours, or mail me at \href{mailto:jrfaudree@alaska.edu}{\texttt{jrfaudree\@@alaska.edu}}.

I will use Canvas to send announcements.  If I need to contact you outside of class times, I'll try to email via Canvas.  Please set the email address in Canvas to one that you check regularly!


%\begin{table}\caption{(tentative) Schedule of Topics}

\begin{center}
\textbf{(tentative) Schedule of Topics}\\

\begin{tabular}{c | c | c || c | c | c}
week & dates &topics \\
\hline \hline
1& 8/26 \& 8/30 &Ch 1 &9& 10/21 \& 10/25 &Ch 11 \\ \hline
2& 9/2 \& 9/6& Ch 2 & 10& 10/28 \& 11/1& Ch 11\\ \hline
3& 9/9 \& 9/13 & Ch 3 & 11& 11/4 \& 11/8& Ch 12, \\
&&&&&Midterm 2\\ \hline
4& 9/16 \& 9/20 & Ch 5 & 12& 11/11 \& 11/15& Null Spaces\\ \hline
5& 9/23 \& 9/27& Ch 6 & 13& 11/18 \& 11/22& Determinants \\ \hline
6& 9/30 \& 10/4& Ch 7, &14& 11/25 \& 11/29&Eigenvalues, \\
&&Midterm 1 &&&Thanksgiving\\ \hline
7& 10/7 \& 10/11& Ch 8 & 15& 12/2 \& 12/5&Eigenvalues\\ \hline
8& 10/14 \& 10/18& Ch 10 & 16& 12/11& Final Exam, \\
&&&&&Wednesday, 8:00am-10:00am\\ \hline
\end{tabular}
\end{center}

\newpage


\textbf{Miscellaneous Other Issues:}

\textbf{Incomplete Grade} 
Incomplete (I) will only be given in DMS courses in cases where the student has completed the majority (normally all but the last three weeks) of a course with a grade of C or better, but for personal reasons beyond his/her control has been unable to complete the course during the regular term. 

\textbf{Late Withdrawals} 
A withdrawal after the deadline (currently 9 weeks into the semester) from a DMS course will normally be granted only in cases where the student is performing satisfactorily (i.e., C or better) in a course, but has exceptional reasons, beyond his/her control, for being unable to complete the course. These exceptional reasons should be detailed in writing to the instructor, department head and dean.

\textbf{Academic Dishonesty}
Academic dishonesty, including cheating and plagiarism, will not
be tolerated.  It is a violation of the Student Code of Conduct
and will be punished according to UAF procedures.

\textbf{The Use of AI Software}
Using a tool like ChatGPT to solve homework problems is not that much different from talking to a classmate or searching for an answer using Google. You will get a response which may or may not be correct. The same rules apply to AI as to any kind of collaboration: You must write your solution independently. To copy and paste an answer from a friend, a book, or an online source would constitute a violation of academic integrity, but more important, it is self-defeating.

\begin{center}\textbf{\large{Official UAF Syllabus Addendum}}\end{center}

\noindent{\bf COVID-19 statement:} Students should keep up-to-date on the university's policies, practices, and mandates related to COVID-19 by regularly checking this website: \url{https://sites.google.com/alaska.edu/coronavirus/uaf?authuser=0}

Further, students are expected to adhere to the university's policies, practices, and mandates and are subject to disciplinary actions if they do not comply.

\noindent{\bf Student protections statement:} UAF embraces and grows a culture of respect, diversity, inclusion, and caring. Students at this university are protected against sexual harassment and discrimination (Title IX). Faculty members are designated as responsible employees which means they are required to report sexual misconduct. Graduate teaching assistants do not share the same reporting obligations. For more information on your rights as a student and the resources available to you to resolve problems, please go to the following site: \url{https://catalog.uaf.edu/academics-regulations/students-rights-responsibilities/}.

\noindent{\bf Disability services statement:} I will work with the Office of Disability Services to provide reasonable accommodation to students with disabilities.

\noindent{\bf Student Academic Support:}
\begin{itemize}
\setlength\itemsep{0em}
        \item Speaking Center (907-474-5470,
        \mailto{uaf-speakingcenter@alaska.edu}, Gruening 507)
\item Writing Center (907-474-5314, \mailto{uaf-writing-center@alaska.edu}, Gruening 8th floor)
\item UAF Math Services, \mailto{uafmathstatlab@gmail.com}, Chapman Building (for math fee paying students only)
\item Developmental Math Lab, Gruening 406
\item The Debbie Moses Learning Center at CTC (907-455-2860, 604 Barnette St, Room 120,\\ \mailto{https://www.ctc.uaf.edu/student-services/student-success-center/})
\item For more information and resources, please see the Academic Advising Resource List (\url{https://www.uaf.edu/advising/lr/SKM_364e19011717281.pdf})
\end{itemize}

\noindent{\bf Student Resources:}
\begin{itemize}
\setlength\itemsep{0em}
\item Disability Services (907-474-5655, \mailto{uaf-disability-services@alaska.edu}, Whitaker 208)
\item Student Health \& Counseling [6 free counseling sessions] (907-474-7043, \url{https://www.uaf.edu/chc/appointments.php}, Whitaker 203)
\item Center for Student Rights and Responsibilities (907-474-7317, \mailto{uaf-studentrights@alaska.edu}, Eielson 110)
\item Associated Students of the University of Alaska Fairbanks (ASUAF) or ASUAF Student Government (907-474-7355, \mailto{asuaf.office@alaska.edu}{asuaf.office@alaska.edu}, Wood Center 119)
\end{itemize}

\noindent{\bf ASUAF advocacy statement} 
The Associated Students of the University of Alaska Fairbanks, the student government of UAF, offers advocacy services to students who feel they are facing issues with staff, faculty, and/or other students specifically if these issues are hindering the ability of the student to succeed in their academics or go about their lives at the university. Students who wish to utilize these services can contact the Student Advocacy Director by visiting the ASUAF office or emailing \mailto{asuaf.office@alaska.edu}{asuaf.office@alaska.edu}

\noindent{\bf Nondiscrimination statement:}
The University of Alaska is an affirmative action/equal opportunity employer and educational institution. The University of Alaska does not discriminate on the basis of race, religion, color, national origin, citizenship, age, sex, physical or mental disability, status as a protected veteran, marital status, changes in marital status, pregnancy, childbirth or related medical conditions, parenthood, sexual orientation, gender identity, political affiliation or belief, genetic information, or other legally protected status. The University's commitment to nondiscrimination, including against sex discrimination, applies to students, employees, and applicants for admission and employment. Contact information, applicable laws, and complaint procedures are included on UA's statement of nondiscrimination available at www.alaska.edu/nondiscrimination. For more information, contact:

\begin{tabular}{l}
UAF Department of Equity and Compliance\\
1760 Tanana Loop, 355 Duckering Building, Fairbanks, AK  99775\\
907-474-7300\\
\mailto{uaf-deo@alaska.edu}
\end{tabular}

\vfill

\quad \hfill \scriptsize syllabus version: \today \normalsize

\end{document}