\documentclass[minion]{homework}
\doclabel{Math F314: Linear Algebra}
\docdate{Homework 12 Determinant Problems}
% \usepackage{cmacros}
\newcommand{\vs}{\mathbf{s}}
\newcommand{\vx}{\mathbf{x}}
\newcommand{\vB}{\mathbf{B}}
\newcommand{\vb}{\mathbf{b}}
\newcommand{\vr}{\mathbf{r}}
\newcommand{\vy}{\mathbf{y}}
\newcommand{\vu}{\mathbf{u}}
\newcommand{\vv}{\mathbf{v}}
\newcommand{\vw}{\mathbf{w}}
\newcommand{\vzero}{\mathbf{0}}

\newcommand{\Reals}{\mathbb{R}}
\def\vectwo#1#2{\begin{bmatrix}#1\\#2\end{bmatrix}}
\def\vecthree#1#2#3{\begin{bmatrix}#1\\#2\\#3\end{bmatrix}}
\def\vecfour#1#2#3#4{\begin{bmatrix}#1\\#2\\#3\\#4\end{bmatrix}}
\begin{document}
\begin{problems}

    


    \problem A $4 \times 4$ matrix has $\det(A)=1/3$.  Find
    $\det(2A)$, $\det(-A)$, $\det(A^2)$ and $\det(A^{-1})$.

    \problem Find two $2\times 2$ matrices $A$ and $B$ with 
    $\det(A)=1$ and $\det(B)=1$ but $\det(A+B)=0$. So there is
    no rule $\det(A+B)=\det(A)+\det(B)$.

    \problem Compute the determinant of 
    \[ A=
      \begin{bmatrix}
        1 & 2 & 3 & 0 \cr
        2 & 6 & 6 & 1 \cr
        -1 & 0 & 0 & 3 \cr
        0 & 2 & 0 & 7 
      \end{bmatrix}
    \]
    by reducing to an upper triangular matrix.

    \problem Compute the determinant of 
    \[B=
      \begin{bmatrix}
        1 & 2 & 3  \cr
        2 & 4 & 10 \cr
        5 & 6 & 7 \cr
      \end{bmatrix}
    \]
    by reducing to an upper triangular matrix.  Note that you will need
    to keep track of row interchanges.

    \problem Find the determinant of 
    \[C=
    \begin{bmatrix} 1 \cr 2 \cr 3\end{bmatrix} 
    \begin{bmatrix} 4 & 5 & 6\end{bmatrix}.
    \]
    Then explain your result.

    \problem Compute the determinant of 
    \[    
    \begin{bmatrix}
        1 & 2 & 3 \cr
        4 & 4 & 4 \cr
        5 & 6 & 7
    \end{bmatrix}
    \]
    by expansion along the third row, i.e. the long formula with six terms 
    coming from three $2\times 2$ determinants.

    \problem
    Compute the determinant of
    $$
    A=\begin{bmatrix}
    5 & 1 & -1 & 2 & 1 \\
    3 & 0 & 0 & 0 & 3  \\
    2 & 3 & 0 & 1 & 2  \\
    0 & 0 & 1 & 0 & 1  \\
    6 & 0 & 0 & 0 & 1  \\
    \end{bmatrix}.
    $$
    You'll want to choose your expansion row wisely...

%    \problem The matrix $E_k$ is a $k\times k$ matrix
%    that is all zeros, except for the entries on or adjacent
%    to the main diagonal, where the values are 1s.  For example,
%    \[
%        E_5 = 
%    \begin{bmatrix}
%        1 & 1 & 0 & 0 & 0\cr
%        1 & 1 & 1 & 0 & 0\cr
%        0 & 1 & 1 & 1 & 0\cr
%        0 & 0 & 1 & 1 & 1\cr
%        0 & 0 & 0 & 1 & 1\cr        
%    \end{bmatrix}
%    \]
%    \begin{enumerate}
%        \item Compute $\det(E_1)$, $\det(E_2)$ and $\det(E_3)$. Note that
%        \[
%            E_1 = \begin{bmatrix} 1 \end{bmatrix}, \quad 
%            E_2 = \begin{bmatrix} 1 & 1 \cr 1 & 1 \end{bmatrix}, \quad 
%            E_3 = \begin{bmatrix} 1 & 1 & 0\cr 1 & 1 & 1 \cr 0 & 1 & 1 \cr \end{bmatrix}.
%        \]
%        \item Show that $\det(E_5) = \det(E_4) - \det(E_3)$ by expanding
%        on the first row. You don't need to compute the number, just show that the formula holds.
%        \item This formula holds generally: $\det(E_{k+1})=\det(E_{k})-\det(E_{k-1})$.
%        Using this fact, compute $E_k$ for $k=3, 4, 5, 6, 7, 8$.
%        \item There's a pattern here! Use the pattern to compute $E_{100}$.
%    \end{enumerate}
%
%    \problem Compute the determinant of the  matrix
%    \[
%    \begin{bmatrix} 2 & -1 & 0 \cr
%        -1 & 2 & -1 \cr
%        0 & -1 & 2 
%    \end{bmatrix}    
%    \]
%    two different ways.  First, reduce to upper trianglular. Second,
%    use expansion along the first row.
%    
%   \problem Consider the vectors $v_1 = (1,2,1,-2)$ and $v_2 = (1,2,3,4)$.
%Let $V$ be the collection of all linear combinations of $v_1$ and $v_2$.
%This is a two dimensional plane in $\Reals^4$.  Let $W$ be the collection
%of all vectors that are perpendicular to all the vectors in $V$.
%This is known as the orthogonal complement of $V$ and is sometimes
%written $W = V^\perp$.
%\begin{enumerate} 
%    \item Show that if $w\in W$ then $v_1^T w=0$ and $v_2^T w = 0$. (This is easy.)
%    \item Show that if $w$ is a vector satisfying the two conditions
%    $v_1^T w=0$ and $v_2^T w = 0$ then in fact $w\in W$. Hint: an
%    arbitrary element in $V$ has the form $v = c_1 v_1 +c_2 v_2$.  Now
%    take some dot products.
%    \item The set of vectors $w$ satisfying $v_1^T w=0$ and $v_2^T w = 0$
%    is the nullspace of a specific matrix.  What is the matrix?
%    \item Determine the nullspace of this matrix to determine all the
%    vectors in $W$.
%\end{enumerate}
\end{problems}

\end{document}
