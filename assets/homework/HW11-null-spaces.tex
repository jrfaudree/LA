\documentclass[minion]{homework}
\doclabel{Math F314: Linear Algebra}
\docdate{Homework 11 Problems}
% \usepackage{cmacros}
\newcommand{\vs}{\mathbf{s}}
\newcommand{\vx}{\mathbf{x}}
\newcommand{\vB}{\mathbf{B}}
\newcommand{\vb}{\mathbf{b}}
\newcommand{\vr}{\mathbf{r}}
\newcommand{\vy}{\mathbf{y}}
\newcommand{\vu}{\mathbf{u}}
\newcommand{\vv}{\mathbf{v}}
\newcommand{\vw}{\mathbf{w}}
\newcommand{\vzero}{\mathbf{0}}

\newcommand{\Reals}{\mathbb{R}}
\newcommand{\bbm}{\begin{bmatrix}}
\newcommand{\ebm}{\end{bmatrix}}
\def\vectwo#1#2{\begin{bmatrix}#1\\#2\end{bmatrix}}
\def\vecthree#1#2#3{\begin{bmatrix}#1\\#2\\#3\end{bmatrix}}
\def\vecfour#1#2#3#4{\begin{bmatrix}#1\\#2\\#3\\#4\end{bmatrix}}
\begin{document}
\begin{problems}

\problem Below is a system of first-order, linear, differential equations. Rewrite this system in the form $Au=\frac{du}{dt}, \: u=u(0)$ for $t=0.$ (That is, you need to define $u$, $A$, and $u(0).$ You are \emph{not} being asked to solve the system.)

\begin{tabular}{lll}
$\frac{dv}{dt}=4v-5w$&\quad& $v=8$ when $t=0$\\
&&\\
$\frac{dw}{dt}=2v-3w$&\quad& $w=5$ when $t=0$\\
\end{tabular}

    \problem Consider the matrix 
    \[
A = \begin{bmatrix}
    1 &2 &2 &4 & 6\cr
    1 &2 &3 &6 &9\cr
    2 &4 &5 &10 &16   
\end{bmatrix}
\]
Find the null space of $A$ by using elementary row operations to produce the reduced row echelon form of $A.$ (You will need to record these for problem 4 later. If you are efficient, you will use a total of 5 row operations.)
 
 \problem For the same matrix $A$ as in the previous problem,
    one solution of $Ax=(1,3,6)$ is $x=(5,-4,3,-2,1)$.  Find
    all solutions to $Ax=(1,3,6).$

 \problem We now return to the matrix $A$ of problem 2. 
	\begin{subproblems}
	\item Demonstrate that if $E_1=\bbm 1&0&0\\-1&1&0\\0&0&1\ebm$, then $E_1A$ is the result of doing an elementary row operation on $A$ that involves leaving rows 1 and 3 unchanged, but replacing the second row of $A$ with $row_2(A)-row_1(A)$. Indeed $E_1$ stands for ``first elementary row operation."
	\item Find $E^{-1}$ (which is easy!). 
	\item Find the matrices $E_1,E_2,E_3,E_4,$ and $E_5$ for each of \textbf{your} row operations from problem 2.
	\item Use a computational tool to find the product $B=E_5E_4E_3E_2E_1$ and verify that $BA$ is $A$ in reduced row echelon form.
	\item Explain why you know that $B$ is invertible.
	\item Let $C=rref(A)$, the reduced row echelon form of $A.$ Write $A$ in terms of $B$ and $C.$ 
	\item Part (f) above is another way of factoring a matrix $A$ (called $LU$-factorization) and it can also be used to solve equations. You don't have to do anything for this part. I put it in so you would know why you were asked to do this.
	\end{subproblems}

%    \problem Find a $3\times 3$ matrix $F$ such that
%    $Fx = (x_1, x_2 - 2 x_1, x_3 + 4 x_1)$.  Also,
%    compute $F^{-1}$ using any technique that seems convenient,
%    including the method of making an educated guess!
%
%    \problem Suppose $A$ is a $3\times 5$ matrix.  Find
%    a $3\times 3$ matrix $E_1$ such that the rows of $E_1A$ 
%    are as follows:
%    \begin{enumerate}
%        \item Row 1 is row 1 of $A$
%        \item Row 2 is row 2 of $A$ minus twice row 1 of $A$.
%        \item Row 3 is row 3 of $A$ plus four times row 1 of $A$
%    \end{enumerate}
%    The matrix $E_1$ is called an elimination matrix.  You should find
%    that $E_1$ is lower-triangular, and that most of the interesting things
%    happen in the first column. Also, compute $E_1^{-1}$.

%    \problem As in the problem above, find a different elimination
%    matrix $E_2$ such that the first two rows of $E_2A$ are those of $A$,
%    but the third row is row three of $A$ minus 6 times row 2 of $A$.
%
%    \problem Ok, let's go back to the matrix $A$ of the first problem:
%    \[
%        A = \begin{bmatrix}
%            1 &2 &2 &4 & 6\cr
%            1 &2 &3 &6 &9\cr
%            2 &4 &5 &10 &17   
%        \end{bmatrix}
%            \]
%    Find an elimination matrix $E_1$ such that $E_1 A$ is the 
%    first step of elimination so that the first column of $A$
%    is cleared below the pivot. Your matrix $E_1$ will have ones on 
%    the diagonal and the only other non-zero entries will be in the 
%    first column.
%
%    \problem Continuing with the previous problem, at this point you will
%    have found that column 2 is already cleared. Yay! Find an elimination
%    matrix $E_2$ that does the work of clearing column three from here,
%    at which point elimination will be done.  That is, $E_2 E_1 A = B$,
%    where $B$ is the echelon form matrix you computed in question 1.
%    
%    \problem From the previous problem, we know that $E_2 E_1 A = B$,
%    where $B$ is in echelon form.  That means that $A = E_2^{-1} E_1^{-1} B$.
%    Compute each of $E_1^{-1}$, $E_2^{-1}$ and the product $E_2^{-1} E_1^{-1}$.
%    The result will be a lower triangular matrix $L$.  There will be ones 
%    on the diagonal, and the entries below the diagonal will be closely 
%    tied to numbers you saw in the process of doing elimination.
%
%    \problem Suppose you want to find a solution of $Ax=(1,3,6)$.
%    One can do this with $QR$ factorization and the pseudo inverse.
%    But here's another way using the factorization $A=L B$ we just formed.
%    \begin{itemize}
%        \item First, solve $Lw = (1,3,6)$ by using forward substitution. Do this!
%        \item Now find a solution of $Bx = w$, for then $Ax = LBx = Lw = (1,3,6)$.
%        To find this solution, extract the three pivot columns of $B$, 
%        and the result will be an upper-triangular matrix $U$.  Great.  Now 
%        use back substitution to solve $U\tilde x = w$.  This will determine
%        the pivot variables in $\tilde x$, and the solution $x$ has its
%        pivot variables from $\tilde x$ and its free variables all zero.
%        Write down the solution $x$ you just found.
%    \end{itemize}        
%
%    
%    \problem Your solution to $Ax=(1,3,6)$ from the previous problem
%    differs from mine, which was $(5,-4,3,-2,1)$.  So the two
%    solutions differ by an element of the nullspace.  Show that
%    this element of the nullspace really was in the nullspace you 
%    identified way back in Problem 1.  The easiest way to do this will be
%    to work with the ``special'' elements of the nullspace you will have
%    identified back in Problem 1 associated with the free variables 
%    $x_2$ and $x_4$.
    
    \problem Suppose $A$ is an $m\times n$ matrix and 
    that $W$ is an invertible $m\times m$ matrix.
    	\begin{subproblems}
	\item Show that the null space of $A$ and the null space of $WA$
    are the same as each other.  (One strategy is to pick a vector in $N(A)$ and show it must be in $N(WA).$ Then reverse that process.)
    	\item What can you conclude
    about the null space of $WA$ 
    if you don't know that $W$ is invertible?
    	\end{subproblems}

\end{problems}

\end{document}
