\documentclass[minion]{homework}
\doclabel{Math F314: Linear Algebra}
\docdate{Homework Chapter 5 Additional Problem}
% \usepackage{cmacros}
\newcommand{\vs}{\mathbf{s}}
\newcommand{\vx}{\mathbf{x}}
\newcommand{\vB}{\mathbf{B}}
\newcommand{\vb}{\mathbf{b}}
\newcommand{\vr}{\mathbf{r}}
\newcommand{\vy}{\mathbf{y}}
\newcommand{\vu}{\mathbf{u}}
\newcommand{\vv}{\mathbf{v}}
\newcommand{\vw}{\mathbf{w}}
\newcommand{\vzero}{\mathbf{0}}

\newcommand{\Reals}{\mathbb{R}}
\def\vectwo#1#2{\begin{bmatrix}#1\\#2\end{bmatrix}}
\def\vecthree#1#2#3{\begin{bmatrix}#1\\#2\\#3\end{bmatrix}}
\def\vecfour#1#2#3#4{\begin{bmatrix}#1\\#2\\#3\\#4\end{bmatrix}}
\begin{document}
\begin{problems}
    \problem Let
    \[
        a= \vecfour1{-1}00,\qquad b=\vecfour0{1}{-1}0
        \qquad c = \vecfour001{-1},
    \]
    \begin{subproblems}
    \item 
        Perform the Gram-Schmidt algorithm on these vectors (in this order) 
        to determine orthonormal 
    vectors $q_1$, $q_2$ and $q_3$.
    \item Write the vector $d = (1,1,1,-3)$ as a linear combination of $q_1$, $q_2$
    and $q_3$.  Recall that because the $q_i$'s are orthonormal, the coefficients
    of the linear combination are given by $q_i^T d$.
    \item From part (b) you know that $d$ is also a linear combination of 
    the vectors $a$, $b$ and $c$.  In fact, this is easy to spot.  Using
    whatever technique you would like, write $d$ as such a linear combination.
    \item If we performed Gram-Schmidt on the collection of vectors $a$, $b$, $c$ and $d$,
    what would have happened? Be specific in your answer.
    \end{subproblems}
\end{problems}

\end{document}
