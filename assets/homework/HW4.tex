\documentclass[minion]{homework}
\doclabel{Math F314: Linear Algebra}
\docdate{Homework 4}
% \usepackage{cmacros}
\newcommand{\vs}{\mathbf{s}}
\newcommand{\vx}{\mathbf{x}}
\newcommand{\vB}{\mathbf{B}}
\newcommand{\vb}{\mathbf{b}}
\newcommand{\vr}{\mathbf{r}}
\newcommand{\vy}{\mathbf{y}}
\newcommand{\vu}{\mathbf{u}}
\newcommand{\vv}{\mathbf{v}}
\newcommand{\vw}{\mathbf{w}}
\newcommand{\vzero}{\mathbf{0}}

\newcommand{\Reals}{\mathbb{R}}
\begin{document}
\begin{problems}
    \problem Let
    \[
        x_1= \begin{bmatrix} 1\\ 2 \\ 1\end{bmatrix},\quad
        x_2= \begin{bmatrix} -3\\ 1 \\ 2\end{bmatrix},\quad
        x_3= \begin{bmatrix} -1\\ 5 \\ 4\end{bmatrix}.
    \]
    Show that $x_1$, $x_2$ and $x_3$ are linearly dependent two different ways:
    \begin{subproblems}
        \item Find coefficients $\beta_1$, $\beta_2$, $\beta_3$ such that
        $\beta_1 x_1 + \beta_2 x_2 +\beta_3 x_3 = 0$. \\
        (There are many ways to do this. You can try just guessing or a more systematic approach.)
        \item Write $x_1$ as a linear combination of $x_2$ and $x_3$.
    \end{subproblems}

    \problem Let
    \[
        y_1= \begin{bmatrix} 1\\ 1 \\ 1\end{bmatrix},\quad
        y_2= \begin{bmatrix} 1\\ 0 \\ 1\end{bmatrix},\quad
        y_3= \begin{bmatrix} 1\\ -1 \\ 2\end{bmatrix}.
    \]
    \begin{subproblems}
        \item Show that $y_1$, $y_2$ and $y_3$ are linearly independent. That is, 
        show that if $\beta_1$, $\beta_2$ and $\beta_3$ are numbers such that
        $\beta_1 y_1 + \beta_2 y_2 + \beta_3 y_3 = 0$ then, in fact, $\beta_1=\beta_2=\beta_3=0$.\\
        (Note that you \emph{must} be systematic about this.)
        \item Briefly explain why $y_1$, $y_2$ and $y_3$ form a basis for $\Reals^3$. Your answer should be one sentence.
        \item Because these vectors form a basis for $\Reals^3$, and because
        $z=(2, 1, 3)$ is a vector in $\Reals^3$, there is a unique linear
        combination $\beta_1 y_1 + \beta_2 y_2 + \beta_3 y_3 = z$.  Find the
        numbers $\beta_1$, $\beta_2$ and $\beta_3$.  \\
        (This will be tedious. We will automate this later. This one time, try doing it by hand. Practice efficiency!)
    \end{subproblems}
    \problem In class I mentioned ``Fact A".  Your book calls this the 
    ``independence-dimension inequality''.  It says ``A linearly independent collection of $n$ vectors has at most $n$ elements''.


    \begin{subproblems}
        \item Consider
        \[
        a_1= \begin{bmatrix} 1\\ 1\end{bmatrix},\quad
        a_2= \begin{bmatrix} 1\\ -1\end{bmatrix},\quad
        a_3= \begin{bmatrix} 1\\ 3 \end{bmatrix}.
        \]
    Explain how you know, without doing any work, that this collection
    is linearly dependent.
    \item Because the collection is linearly dependent, it has redundancy.
    Exhibit this redundancy by finding \emph{three} different linear combinations of 
    the vectors that give you $(0,0)$. \\One of these is trivial!  One
    will take a little bit of work.  Once you have that one, you can easily 
    find infinitely many others, so locating a third will be a breeze!
    \item Exhibit the redundancy differently by finding \emph{three} different 
    linear combinations of $a_1$, $a_2$ and $a_3$ that give you $(4, 7)$.\\
    \textbf{Hint:} Find one linear combination that works.  Then use you answer
    from part \textbf{(a)} to help!
\end{subproblems}

    \problem Suppose $w_1$, $w_2$ and $w_3$ are any vectors at all in $\Reals^{17}$.
    Let $v_1 = w_1-w_2$, $v_2=w_2 - w_3$ and $v_3 = w_3 - w_1$.  Show that
    $v_1$, $v_2$ and $v_3$ are linearly dependent.  \textbf{Hint:} find an explicit
    linear combination that yields zero.

    \problem Text: 5-4

    \problem Text: 5-5 modified as follows.  Suppose $a$ and $b$
    are any $n$-vectors.  Find a formula in terms of $a$ and $b$ 
    for a scalar $\gamma$ such that $a-\gamma b$ is perpendicular to $b$.
    Then draw a picture of $a$, $b$, and $a-\gamma b$ when
    $a=(0,1)$ and $b=(1,1)$.

\end{problems}

\end{document}
