\documentclass[minion]{homework}
\doclabel{Math F314: Linear Algebra}
\docdate{Homework 8 Additional Problem}
% \usepackage{cmacros}
\newcommand{\vs}{\mathbf{s}}
\newcommand{\vx}{\mathbf{x}}
\newcommand{\vB}{\mathbf{B}}
\newcommand{\vb}{\mathbf{b}}
\newcommand{\vr}{\mathbf{r}}
\newcommand{\vy}{\mathbf{y}}
\newcommand{\vu}{\mathbf{u}}
\newcommand{\vv}{\mathbf{v}}
\newcommand{\vw}{\mathbf{w}}
\newcommand{\vzero}{\mathbf{0}}

\newcommand{\Reals}{\mathbb{R}}
\def\vectwo#1#2{\begin{bmatrix}#1\\#2\end{bmatrix}}
\def\vecthree#1#2#3{\begin{bmatrix}#1\\#2\\#3\end{bmatrix}}
\def\vecfour#1#2#3#4{\begin{bmatrix}#1\\#2\\#3\\#4\end{bmatrix}}
\begin{document}
\begin{problems}

    \problem Text: 11.16. You can assume that $A$ is $5\times 5$. And don't bother with the ``Does this make sense'' part of the question.

    \problem Supplemental problem: 11.11. \\
    You can solve this by inspection. Your first step is to determine the dimensions of the matrix right inverse matrix. As you know, there are many solutions.
   % \textbf{Hint:} Consider block multiplication
%    \[
%        \begin{bmatrix} A & a\end{bmatrix} \begin{bmatrix}B \cr 0\end{bmatrix} = \begin{bmatrix}AB + a0\end{bmatrix}
%    \]
%    where $A$ and $B$ are $2\times 2$, $a$ is $2\times 1$ and the $0$ is a $1\times 2$ zero row.


    \problem The matrix 
    \[
    A  =\frac{1}{\sqrt{2}}\begin{bmatrix} 3 & -2 \\ -3 & -4 \end{bmatrix}
    \]
    admits the $QR$ factorization
    \[
    A =   \left(\frac{1}{\sqrt{2}}\begin{bmatrix} 1 & 1 \\ -1 & 1 \end{bmatrix}\right)
    \begin{bmatrix} 3 & 1 \cr 0 & -3\end{bmatrix}
    \]
    You don't need to show this.  Instead, use the $QR$ factorization
    to solve $Ax=b$ with $b=(3,-5)$.  

    Note: For a matrix as small as a $2\times 2$, we wouldn't bother with
    $QR$ factorization. We would simply write down
    the inverse matrix and use it to solve the system.  
    The point of this problem is for you to get a little practice with
    what the the steps of solving the system with $QR$ factorization actually are,
    without having to do an enormous amount of arithmetic.
\end{problems}

\end{document}
